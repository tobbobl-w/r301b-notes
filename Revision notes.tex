\documentclass[12pt]{article}
\usepackage{graphicx}
\usepackage{amsmath}
\usepackage{amssymb}
\usepackage[authoryear]{natbib}
\usepackage{enumerate}
\usepackage{comment}
\setlength{\parindent}{0pt}
\usepackage[parfill]{parskip}
\date{Lent 2023}
\title{R301 Revision notes}
\author{Tobias Leigh-Wood}
\usepackage[letterpaper, margin=1.1in]{geometry}
\linespread{1.1}

\begin{document}
\maketitle

\section*{Plan}
\begin{itemize}
    \item I want a list of all the model definitions and how they relate to each other.
    \item Below this have a description of each one and the key assumptions.

\end{itemize}

\begin{itemize}
    \item probit, logit, control function, reduced form [think there were some others in this category]
    \item Pure conditional logit, Multinomial logit, multinomial probit (trinomial probit), mixed logit
    \item FE, RE, CRE
    \item Bayesian stuff? bayes factor/odds ratio
\end{itemize}

\section*{Binary response models}
\begin{itemize}
    \item Partial Effect: $ \frac{\partial E_{u}[Y(x)]}{\partial x} =  \frac{\partial F(x)}{\partial x}$ this is an random variable associated with x. Two associated objects of interest are:
          Partial Effect at the Average (PEA) -- $\frac{\partial F(x)}{\partial x}\Bigr|_{\substack{x_{e}=E(x)}}$ and
          the Average Partial Effect (APE) -- $E\big[\frac{\partial F(x)}{\partial x}\big]$
    \item Partial effects in probit model with unobserved heterogeneity. $c_{i}$ denotes unobserved heterogeneity.
          \begin{align*}
              P(y_{i} = 1 | x_{i}, c_{i})     & = \Phi(x_{i}\beta + \gamma c_{i}) \\
                                              & = \Phi(x_{i}\beta / \sigma)       \\
              \text{Where} \quad \quad \sigma & =\gamma^{2}\tau^{2} + 1
          \end{align*}
          So there is attenuation bias in the coefficient estimates of $\beta$ since we do not know $\gamma$ or $\tau$.
          We cannot get correct partial effects for this reason.

    \item We can still get the APE since
          \begin{align*}
              E_{c}[\beta\phi(x_{i}\beta + \gamma c_{i})] & = \frac{\beta}{\sigma}
              \phi(\frac{\beta x_{i}}{\sigma})
          \end{align*}
          \textbf{anyway}. And this is what we estimate.
          If there is correlation we will not get consistent estimates of the APE in probit, but we won't in linear models either.
    \item \textbf{Control Functions}
          Generic definition is a variable that when added to a regression renders an endogenous variable exogenous.
          In our setting we view it as an alternative to other IV estimators.

    \item Aside: the model or structural model refers to the equation we are trying to estimate.
          The \textbf{reduced form} tells us how certain variables are linked together.
          For example, if there is a variable that can solve as an instrument it will be in the reduced form.

    \item We consider four situations:
          \begin{table}[htbp]
              \centering\begin{tabular}{ l l l }
                  \textbf{Structural Model} & \textbf{Reduced form} & \textbf{Estimation technique} \\  [0.5ex]
                  Linear                    & Linear                & CF, 2SLS                      \\
                  Linear                    & Nonlinear             & CF                            \\
                  Nonlinear                 & Linear                & CF, 2SLS                      \\
                  Nonlinear                 & Nonlinear             & MLE
              \end{tabular}

          \end{table}

    \item Consider a control function approach to the continuous linear-linear problem.
          $y = x^{e}\beta_{o} + x^{o}\beta_{o} + \epsilon$ and reduced form $x^{e} = z \theta + e $.
          We require 2SLS assumptions.
          Control function approach: estimate a model for the endogenous explanatory variables (EEVs)
          as a function of the instruments -- i.e. the reduced form.
          And then use these residuals in the structural model.
    \item Control function is good because it gets around the forbidden regression for linear structural model
          vs nonlinear reduced form.




\end{itemize}


\end{document}